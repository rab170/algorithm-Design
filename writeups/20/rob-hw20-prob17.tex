 
%
% Homework Details
%   - Title
%   - Due date
%   - Class
%   - Section/Time
%   - Instructor
%   - Author
%


%
% Basic Document Settings
%

\documentclass{article}

\usepackage{fancyhdr}
\usepackage{extramarks}
\usepackage{amsmath}
\usepackage{amsthm}
\usepackage{amsfonts}
\usepackage{tikz}
\usepackage[plain]{algorithm}
\usepackage[noend]{algpseudocode}
\usepackage{amssymb}

\usetikzlibrary{automata,positioning}


\topmargin=-0.45in
\evensidemargin=0in
\oddsidemargin=0in
\textwidth=6.5in
\textheight=9.0in
\headsep=0.25in\newcommand{\hmwkClassTime}{Section A}
\linespread{1.1}

\renewcommand\headrulewidth{0.4pt}
\renewcommand\footrulewidth{0.4pt}
\setlength\parindent{0pt}

%
% Create Problem Sections
%

\newcommand{\enterProblemHeader}[1]{
    \nobreak\extramarks{}{Problem \arabic{#1} continued on next page\ldots}\nobreak{}
    \nobreak\extramarks{Problem \arabic{#1} (continued)}{Problem \arabic{#1} continued on next page\ldots}\nobreak{}
}

\newcommand{\exitProblemHeader}[1]{
    \nobreak\extramarks{Problem \arabic{#1} (continued)}{Problem \arabic{#1} continued on next page\ldots}\nobreak{}
    \stepcounter{#1}
    \nobreak\extramarks{Problem \arabic{#1}}{}\nobreak{}
}

\setcounter{secnumdepth}{0}
\newcounter{partCounter}
\newcounter{homeworkProblemCounter}
\setcounter{homeworkProblemCounter}{1}
\nobreak\extramarks{Problem \arabic{homeworkProblemCounter}}{}\nobreak{}

\newenvironment{homeworkProblem}{
    \section{ }
    %\setcounter{partCounter}{1}
    %\enterProblemHeader{homeworkProblemCounter}
}{
    \exitProblemHeader{homeworkProblemCounter}
}



% 
% Header and Footer definition
%

\pagestyle{fancy}
\lfoot{\lastxmark}
\cfoot{$_{Buck}$ $_{Young}$ $_{and}$ $_{Rob}$ $_{Brown}$}


%
% Title Page
%

\title{
    \vspace{2in}
	\textmd{\textbf{\ClassNumber}} \\
    \textmd{\textbf{\ClassName}} \\    
    \normalsize\vspace{0.1in}\small{\hmwkTitle} \\
    \normalsize\vspace{0.1in}\small{Problems \hmwkProblems} \\
	\normalsize\vspace{0.1in}\small{Due \hmwkDueDate}    \\
    \vspace{3in}
}

\author{\textbf{\hmwkAuthorName}}
\date{}

\renewcommand{\part}[1]{\textbf{\large Part \Alph{partCounter}}\stepcounter{partCounter}\\}






% 	% 	%	%	%	%	%
%	Document Start 		%
% 	% 	% 	% 	% 	% 	%

\begin{document}
\pagebreak

\begin{homeworkProblem}
\centerline{\textbf{Problem 17}}
\leavevmode
\\
 (2 points) The input to the Fixed Hamiltonian path problem is an undirected graph G and two vertices
x and y in G. The problem is to determine if there is a simple path between x and y in G that spans
all the vertices in G. A path is simple if it doesn't include any vertex more than once. Show that
if the Fixed Hamiltonian path problem has a polynomial time algorithm then the Hamiltonian cycle
problem has a polynomial time algorithm.
\\
\begin{algorithmic}[1]
\Function{Ham-Cycle}{G}
	\State $H = G$
	\State $x = v\in G$  \space\space\space//arbitrarily choose a vertex
	\For{$y \in adj(x)$} \space\space\space\space //for each adjacent vertex, check for a Hamiltonian path
		\If{$\textbf{HAM-PATH(H, x, y)}$ == True}
			\State \Return True
		\EndIf
	\EndFor 
	\State \Return False
\EndFunction
\end{algorithmic}
\leavevmode
\\
We are required to inefficiently check every vertex (y) adjacent to x for a Hamiltonian Path from x to y. In vertexes with multiple adjacencies it is not necessarily the case that the Hamiltonian Cycle(s) must use every edge connecting $x$ to $y_i$. This gives way to multiple "path" options to construct the same cycle. Some will work, some will not. Thus we must check all of them. See one example of this below.
\\ \\ \\ \\ \\ \\ \\ \\ \\
\textbf{BONUS:}\\
To circumvent this issue, we add a new node ("y") to our graph H. Create edges connecting y to x and every vertex in its adjacency. This vertex (y) acts as an OR for all of the vertices in x's adjacency. Any Hamiltonian Path constructible from x's adjacency is considered, and the new node does not contribute the the validity of a cycle because we visit it by default. See graph below as an example.

\leavevmode
\\ \\ \\ \\ \\ \\ \\ \\ 
\begin{algorithmic}[1]
\Function{Ham-Cycle}{G}
	\State $x = v\in G$  \space\space\space//arbitrarily choose a vertex
	\State $y =$new vertex with connections to it from x and every vertex in adj(x)
	\State $H = G \cup \{y\}$
\State \Return $\textbf{HAM-PATH(H, x, y)}$
\EndFunction
\end{algorithmic}

\end{homeworkProblem}
\pagebreak

\end{document}
