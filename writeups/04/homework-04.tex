 
%
% Homework Details
%   - Title
%   - Due date
%   - Class
%   - Section/Time
%   - Instructor
%   - Author
%

\newcommand{\hmwkTitle}{Greedy Algorithms}
\newcommand{\hmwkProblems}{9 and 10}
\newcommand{\hmwkDueDate}{September 5, 2014}
\newcommand{\ClassName}{Algorithm Design}
\newcommand{\ClassNumber}{CS 1510}
\newcommand{\hmwkAuthorName}{Buck Young and Rob Brown}



%
% Basic Document Settings
%

\documentclass{article}

\usepackage{fancyhdr}
\usepackage{extramarks}
\usepackage{amsmath}
\usepackage{amsthm}
\usepackage{amsfonts}
\usepackage{tikz}
\usepackage[plain]{algorithm}
\usepackage{algpseudocode}
\usepackage{amssymb}

\usetikzlibrary{automata,positioning}


\topmargin=-0.45in
\evensidemargin=0in
\oddsidemargin=0in
\textwidth=6.5in
\textheight=9.0in
\headsep=0.25in\newcommand{\hmwkClassTime}{Section A}
\linespread{1.1}

\renewcommand\headrulewidth{0.4pt}
\renewcommand\footrulewidth{0.4pt}
\setlength\parindent{0pt}

%
% Create Problem Sections
%

\newcommand{\enterProblemHeader}[1]{
    \nobreak\extramarks{}{Problem \arabic{#1} continued on next page\ldots}\nobreak{}
    \nobreak\extramarks{Problem \arabic{#1} (continued)}{Problem \arabic{#1} continued on next page\ldots}\nobreak{}
}

\newcommand{\exitProblemHeader}[1]{
    \nobreak\extramarks{Problem \arabic{#1} (continued)}{Problem \arabic{#1} continued on next page\ldots}\nobreak{}
    \stepcounter{#1}
    \nobreak\extramarks{Problem \arabic{#1}}{}\nobreak{}
}

\setcounter{secnumdepth}{0}
\newcounter{partCounter}
\newcounter{homeworkProblemCounter}
\setcounter{homeworkProblemCounter}{1}
\nobreak\extramarks{Problem \arabic{homeworkProblemCounter}}{}\nobreak{}

\newenvironment{homeworkProblem}{
    \section{ }
    %\setcounter{partCounter}{1}
    %\enterProblemHeader{homeworkProblemCounter}
}{
    \exitProblemHeader{homeworkProblemCounter}
}



% 
% Header and Footer definition
%

\pagestyle{fancy}
\lhead{\ClassNumber\ - \ClassName}
\chead{\hmwkTitle}
\rhead{Problems \hmwkProblems}
\lfoot{\lastxmark}
\cfoot{\thepage}


%
% Title Page
%

\title{
    \vspace{2in}
	\textmd{\textbf{\ClassNumber}} \\
    \textmd{\textbf{\ClassName}} \\    
    \normalsize\vspace{0.1in}\small{\hmwkTitle} \\
    \normalsize\vspace{0.1in}\small{Problems \hmwkProblems} \\
	\normalsize\vspace{0.1in}\small{Due \hmwkDueDate}    \\
    \vspace{3in}
}

\author{\textbf{\hmwkAuthorName}}
\date{}

\renewcommand{\part}[1]{\textbf{\large Part \Alph{partCounter}}\stepcounter{partCounter}\\}






% 	% 	%	%	%	%	%
%	Document Start 		%
% 	% 	% 	% 	% 	% 	%

\begin{document}

\maketitle

\pagebreak




\begin{homeworkProblem}
\centerline{\textbf{Problem 9}}
\leavevmode
\\ \\
\textbf{(a)}
\\
\textbf{Input:} A number n of skiers with heights $p_1, ... , p_n$, and n skies with heights $s_1, ... , s_n$.
\\
\textbf{Output:} The minimal average difference between skier height $p_i$ and ski height $s_{\alpha(i)}$: $\frac{1}{n} \sum\limits_{i=1}^n |p_i - s_{\alpha(i)}|$
\\
\textbf{Theorem:} The given "minimized height difference first" algorithm MF is correct.
\\ \\
\textbf{Proof:} Consider a "counter-example" algorithm CE to prove MF is incorrect.
\\ \\ Let $T$ be the total height difference (and $\frac{T}{n}$ be the average total height difference),
\\ \& $H_i$ be the height difference between a person $p_i$ and a chosen ski $s_{\alpha(i)}$.
\\ \\ \\ \\ \\ \\ \\ \\ \\ \\ \\ \\ \\ \\ \\ \\ \\ \\ \\ \\ \\ \\ \\ \\ 
Clearly $\frac{T_{MF}}{n}$ $\textgreater$ $\frac{T_{CE}}{n}$ and the problem wants us to minimize $\frac{T}{n}$. 
\\ \\ 
Also -- let OPT be the optimal solution to this problem -- we see that MF $\textless$ CE $\leq$ OPT.
\\ \\
$\therefore$ The given algorithm MF is sub-optimal and not correct by way of this counter-example.
\end{homeworkProblem}

\pagebreak

\begin{homeworkProblem}
\textbf{(b)} 
\\
\textbf{Input:} A number n of skiers with heights $p_1, ... , p_n$, and n skies with heights $s_1, ... , s_n$.
\\
\textbf{Output:} The minimal average difference between skier height $p_i$ and ski height $s_{\alpha(i)}$: $\frac{1}{n} \sum\limits_{i=1}^n |p_i - s_{\alpha(i)}|$
\\
\textbf{Theorem:} The given "shortest-skier to shortest-ski" algorithm SS is correct.
\\ \\
\textbf{Proof:} Assume to reach a contradiction that there exists an input on which SS produces an unacceptable output.
\\ \\
Let SS(I) be the output of SS on input I
\\ \& OPT(I) be the optimal solution which agrees with SS(I) the most number of steps
\\ \\ \\ \\ \\ \\ \\
Let $k$ be the first point in time where the pairing of skier to ski in SS(I) differs from the pairing in OPT(I)
\\ \\
Let OPT'(I) = OPT with $s_{\beta}$ and $s_{\alpha}$ swapped.
\\
Note: $p_\alpha$ could be the same person as $p_\beta$ but they do not have to be
\\ \\ \\ \\ \\ \\ \\ \\ \\
Clearly OPT' is more like SS at time step $k$
\\
Ultimately, we want to show that the average height difference between OPT and OPT' has not gotten worse (in order to prove that OPT' is still an acceptable solution).
\\ \\
First we make a few simple observations: by definition of SS, we know that $p_k$ $\textless$ $p_\beta$ and that $s_\alpha$ $\textless$ $s_\beta$
\\ \\ 
Using this knowledge, we can order all possible arrangements of skies and skiers by height:
\\ \\ \\ \\ \\ \\ \\ \\ \\ \\ \\ \\ \\ 
Thus we have 6 clear cases and for each we must show that the average height difference between OPT and OPT' has not gotten worse. We can do this by using an inequality: $|p_k - s_\alpha| + |p_\beta - s_\beta| \leq |p_k - s_\beta| + |p_\beta - s_\alpha|$ (the height difference of OPT' is $\leq$ the height difference of OPT)
\\ \\
If we can prove this inequality valid for each of our six cases, we will have proven that OPT' is no worse than OPT. We may remove the absolute values in the inequality by simply reordering the terms to always produce a positive value (based on which of the terms is larger on a case-by-case basis)...
\\ \\ \\ \\
\centerline{Case 1 : $p_k \leq p_\beta \leq s_\alpha \leq s_\beta$}
\\ \\ 
\\ \\
\\ \\
\\ \\
\centerline{Case 2 : $s_\alpha \leq p_k \leq p_\beta \leq s_\beta$}
\\ \\ 
\\ \\
\\ \\
\\ \\
\centerline{Case 3 : $p_k \leq s_\alpha \leq p_\beta \leq s_\beta$}
\\ \\ 
\\ \\
\\ \\
\\ \\
\centerline{Case 4 : $p_k \leq s_\alpha \leq s_\beta \leq p_\beta$}
\\ \\ 
\\ \\
\\ \\
\\ \\
\centerline{Case 5 : $s_\alpha \leq s_\beta \leq p_k \leq p_\beta$}
\\ \\ 
\\ \\
\\ \\
\\ \\
\centerline{Case 6 : $s_\alpha \leq p_k \leq s_\beta \leq p_\beta$}
\\ \\ 
\\ \\
\\ \\
\\ \\
Our inequality is valid for every case, thus we have shown that the height difference between OPT and OPT' has not gotten any worse.
\\ \\
$\therefore$ We have reached a contradiction because OPT' is still a valid solution and it agrees with SS for one additional step despite OPT being defined as agreeing for the most number of steps.
\end{homeworkProblem}

\pagebreak

\begin{homeworkProblem}
\centerline{\textbf{Problem 10}}
\leavevmode
\\ \\
\textbf{(SJF)}
\\
\textbf{Input:} A collection of jobs $J_1, ... , J_n$, where the $i$th job is a tuple ($r_i, x_i$) of non-negative integers specifying the release time $r$ and size of the job $x$.
\\
\textbf{Output:} A preemptive feasible schedule on one processor that minimizes the total completion time $\sum\limits_{i=1}^n C_i$
\\
\textbf{Theorem:} That the given "shortest job first" algorithm SJF is correct.
\\ \\
\textbf{Proof:}  Consider a "counter-example" algorithm CE to prove SJF is incorrect.
\\ \\ Let $C$ be the total completion time and $C_i$ be the completion time for job $i$.
\\ \\ \\ \\ \\ \\ \\ \\ \\ \\ \\ \\ \\ \\ \\ \\ \\ \\ \\ \\ \\ \\ \\ \\ 
Clearly $C_{SJF}$ $\textgreater$ $C_{CE}$ and the problem requires us to minimize $C$.
\\ \\ 
Also -- let OPT be the optimal solution to this problem -- we see that SJF $\textless$ CE $\leq$ OPT.
\\ \\
$\therefore$ The given algorithm SJF is sub-optimal and not correct by way of this counter-example.
\end{homeworkProblem}

\pagebreak

\begin{homeworkProblem}
\textbf{(SRPT)}
\\
\textbf{Input:} A collection of jobs $J_1, ... , J_n$, where the $i$th job is a tuple ($r_i, x_i$) of non-negative integers specifying the release time $r$ and size of the job $x$.
\\
\textbf{Output:} A preemptive feasible schedule on one processor that minimizes the total completion time $\sum\limits_{i=1}^n C_i$
\\
\textbf{Theorem:} That the given "shortest remaining processing time" algorithm SRPT is correct.
\\ \\
\textbf{Proof:} 
\\
Suppose, for the sake of reaching a contradiction, that the SRPT algorithm produces unacceptable output for some input I. Let $SRPT(I) = \{ J_{\alpha 1} J_{\alpha 2}, ... J_{\alpha n}\}$ be the scheduling output of SRPT, and let and $OPT(I) = \{ J_{\alpha 1} J_{\alpha 2}, ... J_{\alpha n}\}$ be the output of some optimal algorithm which agrees with SRPT for the most number of steps. Each $J$ $\epsilon$ OUTPUT, $J$ identifies a job that executes for a single unit of time $r$ is the remaining number of time units required for job completion. Each tuple represents a 1 time-unit interval of execution.
\\ \\ 
Consider the output of these two algorithms. Let k be the first point of scheduling disagreement, where SRPT schedules $J_{\alpha k}$ which completes at $J_{\alpha k_C}$ and OPT schedules $J_{\beta k}$ which completes at $J_{\beta k_C}$.



\end{homeworkProblem}

\end{document}
